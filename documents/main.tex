\documentclass[10pt, a4paper]{jsarticle}
\usepackage{amsmath, amssymb, mathtools, amsthm, bm}
%\usepackage[top=25truemm,bottom=25truemm,left=25truemm,right=25truemm]{geometry}

%---- begin -----------------------  ページアンカー  --------------------------------%
\usepackage[dvipdfmx]{hyperref}
\usepackage{pxjahyper}
\hypersetup{
    setpagesize=false,
    bookmarksnumbered=true,
    bookmarksopen=true,
    colorlinks=true,
    linkcolor=black,
    citecolor=black,
    urlcolor=black
}
%----------------------------------  ページアンカー  ---------------------- end -----%


%---- begin --------------------------  タイトル  -----------------------------------%

\title{
    艦隊これくしょん -艦これ- \\ \vspace{3mm}
    \Huge 艦これダメージ式及び逆算式定義改vx.x.x
}
\vspace{3mm}
\author{
    \Large{hedgehog}
    \thanks{{\url{https://twitter.com/hedgehog_hasira}}}
}
\date{\today}


%-------------------------------------  タイトル  ------------------------- end -----%

\begin{document}
\maketitle
\begin{abstract}
あれとかこれとかいろいろあったよねぇ~。
でもこれからはこんな感じでこうしていきたいと思うの。
だからいろいろ書いていくよ。
\end{abstract}

%----  文章構造
%\part{title}	部
%%\chapter{title}	章(jarticle・tarticleクラスには用意されていない)
%\section{title}	節
%\subsection{title}	項(小節)
%\subsubsection{title}	目(小々節)
%\paragraph{title}	段落
%\subparagraph{title}	小段落

\section{ダメージ式}

\[
    ダメージ = \big\lfloor (最終攻撃力 - 最終防御力) \times 弾薬量補正 \big\rfloor
\]

\end{document}